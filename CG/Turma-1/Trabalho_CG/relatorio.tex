\documentclass[12pt, a4paper]{article}
\usepackage[utf8]{inputenc}
\usepackage[T1]{fontenc}
\usepackage{lmodern}
\usepackage[brazil]{babel}
\usepackage{geometry}
\usepackage{graphicx}
\usepackage{listings}
\usepackage{xcolor}
\usepackage{hyperref}

% Geometria da página
\geometry{
    a4paper,
    total={170mm,257mm},
    left=20mm,
    top=20mm,
}

% Estilo para blocos de código
\definecolor{codegray}{rgb}{0.95,0.95,0.95}
\lstset{
    backgroundcolor=\color{codegray},
    basicstyle=\ttfamily\small,
    breaklines=true,
    frame=single,
    rulecolor=\color{black},
    showstringspaces=false,
    tabsize=2,
    captionpos=b,
    keywordstyle=\color{blue},
    stringstyle=\color{red},
    commentstyle=\color{green},
}

% Informações do Título
\title{Relatório de Atividade \\ \large Renderizador de Modelos .OBJ}
\author{
    Nome do Aluno: \underline{\hspace{5cm}} \\
    \vspace{5mm} % Espaço entre nome e matrícula
    Matrícula: \underline{\hspace{5cm}}
}
\date{\today}

\begin{document}

% --- Capa ---
\begin{titlepage}
    \maketitle
    \thispagestyle{empty}
\end{titlepage}

\newpage
\tableofcontents
\newpage

\section{Construindo e Utilizando o Programa}

Esta seção detalha os passos necessários para compilar o programa em um ambiente Linux e como utilizá-lo para carregar diferentes modelos 3D.

\subsection{Compilação do Executável}

O programa foi desenvolvido em C++ e utiliza as bibliotecas OpenGL e GLUT para renderização gráfica. Para compilar o código-fonte (`extractor.cpp`) e gerar o arquivo executável, utilize o seguinte comando no terminal:

\begin{lstlisting}[language=bash]
g++ extractor.cpp -o extractor -lGL -lGLU -lglut
\end{lstlisting}

Este comando irá gerar um arquivo executável chamado `extractor` no mesmo diretório.

\subsection{Carregando Modelos}

Para carregar um modelo 3D diferente, é necessário modificar o arquivo-fonte `extractor.cpp`. A chamada para o carregamento do modelo está localizada na função `init()`.

\textbf{Exemplo:}
Para carregar o modelo `teapot.obj`, a linha de código correspondente deve ser:
\begin{lstlisting}[language=C++]
loadModel("./.src/teapot.obj");
\end{lstlisting}

Para carregar o modelo `dragon.obj`, a linha deve ser:
\begin{lstlisting}[language=C++]
loadModel("./.src/dragon.obj");
\end{lstlisting}

Após modificar o arquivo, é necessário recompilar o programa seguindo o passo 1.1 para que as alterações tenham efeito. Os arquivos de modelo (`.obj`) devem estar localizados no diretório `.src/`.

\section{Modelos Utilizados}

Durante o desenvolvimento e depuração do programa, diversos modelos foram testados para garantir a funcionalidade do renderizador. O programa demonstrou sucesso ao carregar, centralizar, escalar e renderizar modelos que seguem o padrão de formato `.obj` e que possuem geometria válida.

Modelos que foram carregados com sucesso incluem:
\begin{enumerate}
    \item \textbf{`teapot.obj`} : O modelo clássico de bule de chá (Utah teapot), que foi renderizado corretamente após a implementação da lógica de carregamento e auto-escala.
    \item \textbf{`[Nome do Modelo 2]`} : (Espaço para descrever outro modelo testado).
    \item \textbf{`[Nome do Modelo 3]`} : (Espaço para descrever outro modelo testado).
\end{enumerate}

O modelo `dragon.obj` apresentou problemas de renderização que não foram resolvidos durante os testes. A causa provável está relacionada a dados corrompidos, geometria não-padrão (polígonos complexos, não-convexos) ou outra inconsistência no próprio arquivo de dados, uma vez que o renderizador se mostrou funcional com o modelo `teapot.obj`.

\section{Recursos Adicionais}

Para a correta compilação e execução do programa, os seguintes recursos são necessários:
\begin{itemize}
    \item \textbf{Compilador C++:} Um compilador C++ padrão, como o `g++`.
    \item \textbf{Bibliotecas Gráficas:} As bibliotecas de desenvolvimento para OpenGL e GLUT. Em sistemas baseados em Debian (como Ubuntu), elas podem ser instaladas com o seguinte comando:
    \begin{lstlisting}[language=bash]
sudo apt-get install freeglut3-dev
    \end{lstlisting}
    Em sistemas baseados em Arch Linux, o comando seria:
    \begin{lstlisting}[language=bash]
sudo pacman -S freeglut
    \end{lstlisting}
    \item \textbf{Arquivos de Modelo:} Os arquivos de modelo no formato `.obj` que se deseja renderizar. Por convenção do projeto, estes devem ser colocados no diretório `./.src/`.
\end{itemize}

\section{Informações Adicionais}
\begin{itemize}
    \item \textbf{Processo de Depuração:} O desenvolvimento foi iterativo, partindo da correção de uma tela preta inicial. As causas investigadas e corrigidas incluíram: (1) ausência de código de desenho na função `display`, (2) configuração incorreta de matrizes e iluminação na função `init`, e (3) implementação de uma rotina de auto-escala e centralização baseada na Bounding Box do modelo para garantir que o objeto estivesse visível na câmera.
    \item \textbf{Compatibilidade de Ambiente:} O programa foi testado em um ambiente Arch Linux com o compositor Hyprland (Wayland). A renderização com GLUT funcionou corretamente após as correções no código, indicando que os problemas iniciais não estavam relacionados ao ambiente, mas a erros de implementação e, posteriormente, a dados de modelo específicos.
    \item \textbf{Limitações e Melhorias Futuras:} O renderizador atual utiliza o modo imediato do OpenGL (`glBegin`/`glEnd`), que é uma abordagem antiga e com baixo desempenho para modelos com alta contagem de polígonos, como o `dragon.obj` (mais de 870.000 faces). Uma futura melhoria seria migrar para o modo retido (moderno) utilizando \textit{Vertex Buffer Objects} (VBOs) para um desempenho de renderização drasticamente superior.
\end{itemize}

\end{document}